\Section{Linear Transformations}{}

\subsection{Basic Notions}

\begin{definition}
    Let \( V, W \) be \( \F \)-vector spaces. A function \( T : V \to W \) is 
    called a linear transformation if for all \( \alpha \in \F, \vec{x}, \vec{y} \in V \)
    \begin{itemize}
        \item \( T(\vec{x} + \vec{y}) = T(\vec{x}) + T(\vec{y}) \), and
        \item \( T(\alpha \vec{x}) = \alpha T(\vec{x}) \)
    \end{itemize}
    Equivalently, \( T \) is a linear transformation if it can be sent through
    any linear combination of two vectors (this is less work to prove
    sometimes).
\end{definition}

\begin{definition}
    Let \( \mathcal{B} = (\vec{v}_1, \ldots, \vec{v}_n) \) be a basis of \( V
    \). Define \(
    [\cdot]_{\mathcal{B}} : V \to \F^n \) via
    \[
        [\vec v]_{\mathcal{B}} = \begin{pmatrix}
            \alpha_1 \\ \alpha_2 \\ \vdots \\ \alpha_n
        \end{pmatrix}
    ,\]
    where \( \alpha_1, \ldots, \alpha_n \) are the unique coefficients such that
    \[
        \vec{v} = \sum_{i = 1}^{n} \alpha_i \vec{v}_i
    .\]
\end{definition}

\begin{theorem}
    Let \( \mathcal{B} = (\vec{v}_1, \ldots, \vec{v}_n) \) be a basis of \( V
    \). Then, \( [\cdot]_{\mathcal{B}} : V \to \F^n \) is a linear
    transformation.
\end{theorem}

\begin{proof}
    Write everything as linear combinations of the basis elements.
\end{proof}

\begin{theorem}
    Let \( V, W \) be \( \F \)-vector spaces. Let \( \mathcal{B} = (\vec{v}_1,
    \ldots, \vec{v}_n) \) be a basis of \( V \), and let \( \vec{w}_1, \ldots,
    \vec{w}_n \in W \). There exists a unique linear transformation \( T : V
    \to W \) such that \( T(\vec{v}_i) = \vec{w}_i \) for all \( i \in \{1, \ldots, n\} \).
\end{theorem}

\begin{proof}
    We shall prove existence first. Indeed, let \( \vec{v} \in V \). We may write
    \[
        \vec{v} = \sum_{i = 1}^{n} \alpha_i \vec{v}_i
    ,\]
    for some \( \alpha_i \in \F \). Then, define
    \[
        T(\vec{v}) = \sum_{i = 1}^{n} \alpha_i \vec{w}_i
    .\]
    It is easy to check that this is linear. We shall now show uniqueness.
    Indeed, let \( S, T : V \to W \) satisfy the above. For any \( \vec{v} \)
    we can write it in the basis like above to get that
    \[
        T(\vec{v}) = \sum_{i = 1}^{n} \alpha_i \vec{w}_i = \sum_{i = 1}^{n} \alpha_i S(\vec{v}_i) = S \left( \sum_{i = 1}^{n} \alpha_i \vec{v}_i  \right) = S(\vec{v})
    .\]
    Thus, \( S = T \).
\end{proof}
