\documentclass[a4paper, 12pt]{article}

\usepackage{chirpstyle}

\begin{document}

\section*{Midterm 1}

Problem \( 5 \) is false guys.

\begin{blackbox}
    \begin{problem}[4]
        Let \( A \) be an \( m \times n \) matrix over \( \F \) whose columns span \( \F^m \). Prove that \( T_A \) is surjective.
    \end{problem}
\end{blackbox}

\begin{proof}
    Always go back to definitions. We want \( T_A \) to be surjective so, given
    any \( \vec{y} \in \F^m \), we want a \( \vec{x} \in \F^n \) so that \( A
    \vec{x} = \vec{y} \).
\end{proof}

\begin{blackbox}
    \begin{problem}
        \( T : V \to W, S : W \to X \) linear transformations. \( T \) is surjective and \( S \) is invertible. Must \( S \circ T \) be surjective? Must \( S \circ T \) be injective?
    \end{problem}
\end{blackbox}

\begin{solution}
    Just go back to definitions.

    \( S \circ T \) is surjective. Let \( \vec{x} \in X \). We shall show that there is some vector \( \vec{v} \in V \) such that \( (S \circ T)(\vec{v}) = \vec{x} \). Since \( S \) is bijective, there exists some \( \vec{w} \in W \) such that \( S(\vec{w}) = \vec{x} \). Since \( T \) is surjective, there exists some \( \vec{v} \in V \) such that \( T(\vec{v}) = \vec{w} \). Thus, \( S(T(\vec{w})) = \vec{x} \).

    \( S \circ T \) is not necessarily injective. Take \( V = \R \) and \( W = \{ \vec{0} \} \). Let \( T(\vec{x}) = 0 \) for all \( \vec{x} \) so \( T \) is surjective. Take \( S \) to be the identity. Then we have that \( S(T(1)) = S(T(2)) = 0 \).
\end{solution}

\begin{blackbox}
    \begin{problem}[8]
        Let \( T : V \to W \) be a bijective linear transformation. Let \( S : W \to V \) be a left-inverse for \( T \). Prove that \( S \) is also a right-inverse for \( T \).
    \end{problem}
\end{blackbox}

\begin{solution}
    Since \( T \) has a right inverse, call it \( R \).
    \[
        R = (S \circ T) \circ R = S \circ (T \circ R) = S
    .\]
    Boom
\end{solution}



\end{document}
