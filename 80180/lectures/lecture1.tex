\Section{Lecture 1}{08-26-25}

Consider the following thought experiment: You're sitting on a train and you
overhear a couple talking behind you in a foreign language. After an hour of
listening intently, you're still not able to pick out and discern what they're
saying.

\textbf{What elements of language may have inhibited you from learning it?} There are actually quite a range of factors that may separate your language from the couple's foreign language:
\begin{multicols}{2}
    \begin{itemize}
        \item Foreign sounds and articulations
        \item Lack of/abstraction away from context
        \item Cultural differences
        \item Grammar differences (ex. subject object verb order in Japanese)
        \item No feedback loop (because you're not in the conversation)
        \item Personal individual quirks of speech that don't reflect the
            ``correct'' way of speaking the language
        \item Accents/dialects
        \item Speaking speed
        \item Same sounds can have different values
    \end{itemize}
\end{multicols}

In an attempt to classify the features that make language what it is, a man
named Charles Hockett in the 1960s gave a list of design features in an article
published to \textit{Scientific American} that were in a sense to be thought of
as a linguistic analogue to Euclid's postulates. Although Hockett laid these
characteristics out because he was interested in how human communication
differed from animal communication, they still have some relevance in the
modern day.

\textbf{Design Features:}

\vspace{-0.2cm}

\begin{enumerate}
    \item \textbf{Vocal Auditory Channel:} Speaking and hearing are the vector of human language (this does not take into account writing and ASL, but we shall forego this)
    \item \textbf{Broadcast Transmission and Directional Reception:} Sound travels in all directions, and humans can hear the direction from which it is coming from.
    \item \textbf{Rapid Fading or Transitoriness:} Sound disappears with the passage of time.
    \item \textbf{Interchangeability:} Humans can mimick (send and receive) any
        signal that is expressed in language. A girl can say "I'm a boy" even
        if they don't identify as such. This is not the case with bees, where
        only the queen bee can communicate with certain pheromones that they
        are the queen bee.
    \item \textbf{Total Feedback:} Speakers can hear, control, and modify their
        own speech in real time.
    \item \textbf{Specialization:} The information conveyed through language is
        deliberate. A dog panting can signal to a human that it feels
        overheated, but this is a byproduct of biology and not an intentional
        response on the dog's part.
    \item \textbf{Semanticity:} Certain sounds can code directly to certain meanings.
    \item \textbf{Arbitrariness:} Language is made up of iconic and arbitrary
        symbols: some language is iconic in the sense that the words are very
        much like what they are describing (example: onomatopoeia,
        counterexample: Japanese onomatopoeia)
    \item \textbf{Discreteness:} We classify sounds into discrete, categorical
        units as opposed to continuous stretches of sound.
    \item \textbf{Displacement:} Language can describe that which is far away
        in location/time/reality (hypotheticals). Human language is not limited
        to the current location and time.
    \item \textbf{Productivity:} Language can create novel sentences and
        express novel ideas.
    \item \textbf{Traditional Transmission:} Language is transmitted and taught
        by speakers of the language.
    \item \textbf{Duality of Patterning:} Messages are created from words and morphemes (which have individual meaning), which are created from sounds and phonemes (which do not have meaning on their own).
\end{enumerate}

\begin{definition}
    \textit{Synchronic linguistics} studies languages at a fixed point in time. \MarginComment{This is sort of a misrepresentation.}
\end{definition}

\begin{definition}
    \textit{Diachronic linguistics} studies how language changes over time.
\end{definition}

\begin{definition}
    \textit{Captured language} is sound that is captured. \MarginComment{???}
\end{definition}

\begin{definition}
    Inferred language is language that is analyzed. \MarginComment{???}
\end{definition}

\begin{definition}
    \textit{e-language} is is the external, observal language and system that a community shares.
\end{definition}

\begin{definition}
    \textit{i-language} is the internal representation and knowledge of a
    language within a human's mind.
\end{definition}

New inventions, political changes, and cultural changes give impetus for
language to change and adapt. This is aided by arbitrariness. Language's
prehistoric development, however, was likely aided by and dependended heavily
upon displacement. One common theory for how displacement developed was that we
started with only being able to process the current moment and then our brains
grew bigger. There are, however, many points that go against this. A (probably
wrong) different theory is that humans needed to communicate that food was in
another area, and thus displacement was born.

This reflects a common theme in language, that humans invent language by
necessity.
