\Section{Basics of Analysis}{}

\subsection{Metric Spaces}

For now, we shall work over the real numbers unless otherwised specified. From context, we can 

\begin{definition}
    A pair \( (X, d) \) is called a \textit{metric space} (where \( X \)
    a set and \( d : X \times X \to \R_{\ge 0} \)) if
    \begin{enumerate}
        \item \( d(x, y) = 0 \) iff \( x = y \),
        \item \( d(x, y) = d(y, x) \), and
        \item \( d(x, z) \le d(x, y) + d(y, z) \).
    \end{enumerate}
\end{definition}

\begin{definition}
    A pair \( (X, \| \cdot \|) \) (where \( X \) is a set and \( \| \cdot \| :
    X \times \R_{\ge 0} \)) is a \textit{normed space} if
    \begin{enumerate}
        \item \( X \) is a vector space (over the reals),
        \item \( \| \alpha x \| = |\alpha| \| x \| \),
        \item \( \| x + y \| \le \| x \| + \| y \| \), and
        \item \( \| x \| = 0 \) iff \( x = 0_X \).
    \end{enumerate}
\end{definition}

Note that every normed space \( (X, \| \cdot \|) \) induces a metric space \(
(X, d) \) where \( d(x, y) := \| y - x \| \).

\begin{example}
    \( \R^d \) with the standard Euclidean metric forms a normed space (and
    thus a metric space).
\end{example}

\begin{definition}
    Pick \( p \in (0, \infty) \). Then, define the \textit{\( L^p \) norm},
    \( | \cdot |_p \), of a vector
    \( x \in \R^d \) via
    \[
        |x|_p := \left( \sum_{i = 1}^{d} |x_i|^p \right)^{1/p}
    ,\]
    and define the \textit{\( L^{\infty} \) norm}, \( |\cdot|_\infty \), via
    \[
        |x|_\infty := \max_{1 \le i \le d} |x_i|
    .\]
\end{definition}

\begin{definition}
    Let \( p, q \in [1, \infty] \). We say that \( p, q \) are \textit{Holder conjugates} if
    \[
        \frac{1}{p} + \frac{1}{q} = 1
    .\]
\end{definition}

\begin{blackbox}
\begin{lemma}
    (Young's Inequality) Let \( a, b \ge 0 \). Then, if \( p, q \) are Holder conjugates,
    \[
        ab \le \frac{a^p}{p} + \frac{b^q}{q}
    .\]
\end{lemma}
\end{blackbox}

\begin{proof}
    This trivially holds if any of \( x, y \) are zero, so assume \( a, b > 0
    \). Observe that \( x \mapsto \log x \) is concave, so by Jensen's,
    \[
        \log \left( \frac{1}{p} a^p + \frac{1}{q} b^q \right) \ge \frac{1}{p} \log(a^p) + \frac{1}{q} \log(b^q) = \log(ab)
    .\]
\end{proof}

The above lemma holds more generally, but we don't need this.

\begin{blackbox}
\begin{lemma}
    (Holder's inequality) Let \( x, y \in \R^d \), and suppose \( p, q \) are
    Holder conjugates. Then,
    \[
        |x \cdot y| \le |x|_p |y|_q
    .\]
\end{lemma}
\end{blackbox}

\begin{proof}
    This holds trivially if any of \( x, y \) are the zero vector, so assume
    this is not the case. Let \( u, v \) denote the vectors formed by taking
    the absolute values of the coordinates of \( x, y \) respectively. It
    follows from the triangle inequality that \( |x \cdot y| \le |u \cdot v|
    \). Moreover, the respective norms of each vector are preserved. Then, we
    have that
    \[
        \frac{u_i}{|u|_p} \cdot \frac{v_i}{|v|_q} \le \frac{|u_i|^p}{p|u|^p_p} + \frac{|v_i|^q}{q|v|^q_q}
    .\]
    Summing over \( i \in \{1, \ldots, d \} \) yields
    \[
        \frac{|u \cdot v|}{|u|_p |v|_q} \le \frac{1}{p} + \frac{1}{q} = 1
    .\]
    Thus, \( |u \cdot v| \le |u|_p |v|_q \), so \( |x \cdot y| \le |x|_p |y|_q \).
\end{proof}

\begin{blackbox}
\begin{theorem}
    \( (\R^d, |\cdot|_p) \) is a normed space iff \( p \ge 1 \).
\end{theorem}
\end{blackbox}

\begin{proof}
    It suffices to show that the triangle inequality holds (the other
    conditions are relatively trivial). It suffices to assume \( x + y \) is
    nonzero (otherwise the inequality follows trivially). Observe that for all
    \( x, y, z \in \R^d \),
    \[
        |(x + y) \cdot z| \le |x \cdot z| + |y \cdot z| \le (|x|_p + |y|_p) |z|_q
    ,\]
    by triangle inequality and Young's inequality. Choose \( z \) such that \(
    z_i = (x_i + y_i)^{p - 1} \). Then, we have that
    \[
        |(x + y) \cdot z| = |x + y|_p^p
    .\]
    Moreover, since \( 1/p + 1/q = 1 \), we have that
    \[
        |z|_q = |x + y|_p^{p/q} = |x + y|_p^{p - 1}
    .\]
    Combining these, we get that
    \[
        |x + y|_p^p \le (|x|_p + |y|_p) |x+y|_p^{p-1}
    ,\]
    and so \( |x + y|_p \le |x|_p + |y|_p \).
\end{proof}
